\documentclass{article}
\pagestyle{empty}

\usepackage{amsmath}
\usepackage{mathspec}
\usepackage{nicefrac}
%fontspec
\setmainfont%[SmallCapsFont={Linux Biolinum Capitals O}]%
{TeX Gyre Pagella}
\def\paragriego{Anaktoria}
% \def\paragriego{GFS Complutum}
\setmonofont{Terminus (TTF)}
%           [BoldFont={TerminusTTF-Bold}]
\setmathsfont(Digits,Latin,Greek){Neo Euler}
\usepackage[a4paper, text={15.5cm, 25cm}]{geometry}
\addtolength\parskip{3mm}

\usepackage{amssymb}
\usepackage{pagecolor}
\usepackage{pbox}
\usepackage{graphicx}
\usepackage{fancyvrb}
\usepackage[lined,spanish]{algorithm2e}
\setlength\parindent{0pt}

\usepackage[hang,flushmargin]{footmisc}

\newcommand\id{\mbox{indeg}}
\newcommand\od{\mbox{outdeg}}

\newcommand\N{\mathbb{N}}
\newcommand\Z{\mathbb{Z}}
\newcommand\Q{\mathbb{Q}}

\newcommand\ret{↵\ } % ⏎, 〈RET〉

\newcommand\tr{^{\mbox{\footnotesize t}}}   % transpuesta

\definecolor{morado}{RGB}{85, 39, 107}
\definecolor{naranja_muy}{RGB}{255, 156, 59} % (1, .612, .231)
\definecolor{naranja_osc}{RGB}{255, 191, 128} % (1, .749, .501)
\definecolor{naranja_claro}{RGB}{255, 229, 204} % (1, .898, .8)
\definecolor{marca}{RGB}{242, 140, 113}

\newcommand\inp[1]{\colorbox{gray!40}{#1}} % ¿Sobra espacio a la izquierda?
\newcommand\mrc{\color{marca!90!black}}

\usepackage{stmaryrd}       % flecha

%% \usepackage{marginnote}
%% \reversemarginpar

\newcounter{pregunta}
\newcommand\prop[1]%
{\addtocounter{pregunta}{1}
\noindent%\marginnote
{\color{naranja_muy}\small\bf\thepregunta)}
%{$\rightarrowtriangle$}\ %
\parbox[t]{.95\linewidth}{\it #1}}

\usepackage[colorlinks=true, urlcolor=blue]{hyperref}

\usepackage{array}
% alineada a la derecha, anchura fija
\newcolumntype{R}[1]{>{\raggedright\arraybackslash}m{#1}}
\newcolumntype{C}{>{$}c<{$}}
\newcolumntype{M}[1]
{>{\centering\arraybackslash $}m{#1}<{$}}

% Para las flechas en tablas...
\usepackage{tikz}
\usetikzlibrary{tikzmark}
\usetikzlibrary{arrows}

\newcommand\aparte[1]{%
\hfill%{\color{black!60}\vline}\quad
\begin{minipage}[t]{.9\linewidth}%
\color{black!80!blue}\small #1
\end{minipage}}

\def\adentro#1{\hspace{.05\linewidth}
  \parbox{.8\linewidth}{\small #1}}

\XeTeXdashbreakstate = 0

\usepackage{xstring}
\usepackage{ifthen}
\usepackage{xifthen}

\usepackage{tcolorbox}
\newtcolorbox{resalte}[1][]%
{coltitle=black,
arc=3pt, colback=naranja_claro,
top=3mm, bottom=3mm, left=3mm, right=3mm,
colframe=naranja_osc, boxrule=1pt,
left skip=0pt, right skip=4cm, #1}
%segmentation style={solid}}
%segmentation engine=empty}
%segmentation style={double=white}}

\newtcolorbox{resalte_chico}[1][]%
{arc=3pt, colback=naranja_claro,
top=1mm, bottom=1mm, left=1mm, right=3mm,
colframe=naranja_osc, boxrule=1pt,
left skip=0pt, right skip=4cm, #1}

\newtcbox{resaltelinea}[1][]%
{tcbox raise base, nobeforeafter,
coltitle=black,
arc=3pt, colback=naranja_claro,
top=1mm, bottom=1mm, left=1mm, right=0pt,
colframe=naranja_osc, boxrule=1pt,
#1}

\newtcbox{resalteajustado}[1][]%
{tcbox raise base, nobeforeafter,
coltitle=black,
arc=3pt, colback=naranja_claro,
top=0mm, bottom=0mm, left=0pt, right=0pt,
colframe=naranja_osc, boxrule=1pt,
#1}

\newcommand\respuestas{{\color{naranja_muy}\bf Respuestas}
\vspace{5mm}}

\newcommand\figs{./figs}
\newcommand\cod{./scripts}
\newcommand\grf{./grafos}

\newcommand\espacio{␣}
\newcommand\vacio{\mbox{\fontspec{FreeSerif}∅}}

\pagecolor{morado!10} % 238, 233, 240 (.933, .914, .941)

\def\truca
{~\par}
\def\trucavuelve
{~\par\vspace{-\baselineskip}}

\usepackage[spanish]{babel}
\usepackage{colortbl}

\newcommand\gr{\color{gray}}

\usepackage{diagbox}
\usepackage{scalerel}
\usepackage{multirow}

\usepackage{eso-pic}

\def\msi{\color{naranja_muy}{\fontspec{FreeSerif}✓}}
\def\mno{{\fontspec{FreeSerif}✗}}

\def\pyt{{\color{naranja_muy}\texttt{>>>}}}
\def\pyc{{\color{naranja_muy}\texttt{...}}}
\def\sh{{\color{naranja_muy}\texttt{\$}}\hspace{-1em} }
\def\shc{{\color{naranja_muy}\texttt{>}}}
\def\sg{{\color{naranja_muy}\texttt{sage:}}}
\def\sgc{{\color{naranja_muy}\texttt{....:}}}
\def\bl#1{{\color{black!50!blue}\texttt{#1}}}
\def\ro#1{{\color{red!80!black}\texttt{#1}}}

\begin{document}
\shorthandoff{>}\shorthandoff{<}
\centerline{\large\bf Particiones}
\vspace{5mm}

La función $p(n)$ cuenta,
%\footnote{\href{http://oeis.org/A000041}{\texttt{http://oeis.org/A000041}}}
para un número natural $n$, de cuántas maneras puede
escribirse como suma de enteros positivos, sin tener en
cuenta el orden de los sumandos.

Por ejemplo, hay $p(5) = 7$ particiones del número $5$:

\begin{minipage}[t]{.13\linewidth}
\begin{center}
{\footnotesize $5$}\\[3mm]

\includegraphics[scale=.65]{\figs/fig_03_01}
\end{center}
\end{minipage}\hfill
\begin{minipage}[t]{.13\linewidth}
\begin{center}
{\footnotesize $4 + 1$}\\[3mm]

\includegraphics[scale=.7]{\figs/fig_03_02}
\end{center}
\end{minipage}\hfill
\begin{minipage}[t]{.13\linewidth}
\begin{center}
{\footnotesize $3 + 2$}\\[3mm]

\includegraphics[scale=.65]{\figs/fig_03_03}
\end{center}
\end{minipage}\hfill
\begin{minipage}[t]{.13\linewidth}
\begin{center}
{\footnotesize $3 + 1 + 1$}\\[3mm]

\includegraphics[scale=.65]{\figs/fig_03_04}
\end{center}
\end{minipage}\hfill
\begin{minipage}[t]{.13\linewidth}
\begin{center}
{\footnotesize $2 + 2 + 1$}\\[3mm]

\includegraphics[scale=.65]{\figs/fig_03_05}
\end{center}
\end{minipage}\hfill
\begin{minipage}[t]{.13\linewidth}
\begin{center}
{\footnotesize $2 + 1 + 1 + 1$}\\[3mm]

\includegraphics[scale=.65]{\figs/fig_03_06}
\end{center}
\end{minipage}\hfill
\begin{minipage}[t]{.13\linewidth}
\begin{center}
{\footnotesize $1 + 1 + 1 + 1 + 1$}\\[3mm]

\includegraphics[scale=.65]{\figs/fig_03_07}
\end{center}
\end{minipage}

\newcommand\mo{\cellcolor{morado!20}}
\newcommand\ma{\cellcolor{morado!40}}  % (187, 169, 196)
\newcommand\me{\cellcolor{naranja_claro}}
\newcommand\mi{\cellcolor{naranja_osc}}

{\renewcommand\arraystretch{1.5}
\begin{tabular}
{R{1cm}|
M{.7cm}|M{.7cm}|M{.7cm}|M{.7cm}|M{.7cm}|M{.7cm}|M{.7cm}|M{.7cm}|M{.7cm}|M{.7cm}|M{.7cm}|C}
$n$ &\gr 0 &\gr 1 &\gr 2 &\gr 3 &\gr 4 &\gr 5 &\gr 6 &\gr 7&\gr 8&\gr 9&\gr 10&\gr\cdots\\ \hline \rowcolor{morado!20}
$p(n)$ & 1 & 1 & 2 & 3 & 5 & \ma 7 & 11 & 15 & 22 & 30 & 42 &\cdots\\
\end{tabular}

\href{http://oeis.org/A000041}
{\texttt{http://oeis.org/A000041}}

\aparte{El valor $p(0) = 1$ proviene de la toma en
consideración de la partición vacía (la que no tiene ningún
sumando) para el número
$0 = \sum(\vacio) = \sum_{i \in \vacio} i$.}

\begin{resalte}
¿Cómo calculamos $p(n)$ a partir de $n$?
\end{resalte}
La estrategia que proponemos comienza haciendo más difícil
el problema, generalizándolo a la evaluación de la función
$p_k(n)$, definida como el número de particiones de $n$ que
tienen $k$ partes ($k$ sumandos) como mucho.

\begin{center}
\begin{minipage}{12cm}
{\renewcommand\arraystretch{1.5}
\begin{tabular}
{c|%{M{.7cm}|
M{.7cm}|M{.7cm}|M{.7cm}|M{.7cm}|M{.7cm}|M{.7cm}|M{.7cm}|M{.7cm}|M{.7cm}|}
\diagbox[height=1cm]{\gr $n$}{\gr $k$} &
\gr 0 &\gr 1 &\gr 2 &\gr 3 &\gr 4 &\gr 5 &\gr 6 &\gr 7&\gr 8\\ \hline
\gr $0$ & \mo 1 & \mo 1& \mo 1& \mo 1& \mo 1& \mo 1& \mo 1& \mo 1& \mo 1\\ \hline
\gr $1$ &  & \mo 1 & \mo 1& \mo 1& \mo 1& \mo 1& \mo 1& \mo 1& \mo 1\\ \hline
\gr $2$ &  & \mo 1 & \mo 2 & \mo 2& \mo 2& \mo 2& \mo 2& \mo 2& \mo 2\\ \hline
\gr $3$ &  & \mo 1 & \mo 2 & \mo 3 & \mo 3& \mo 3& \mo 3& \mo 3& \mo 3\\ \hline
\gr $4$ &  & \mo 1 & \mo 3 & \mo 4 & \mo 5 & \mo 5& \mo 5& \mo 5& \mo 5\\ \hline
\gr $5$ &  & \mo 1 & \ma 3 & \mo 5 & \mo 6 & \mo 7 & \mo 7& \mo 7& \mo 7\\ \hline
\gr $6$ &  & \mo 1 & \mo 4 & \mo 7 & \mo 9 & \mo 10 & \mo 11 & \mo 11& \mo 11\\ \hline
\gr $7$ &  & \mo 1 & \mo 4 & \mo 8 & \mo 11 & \mo 13 & \mo 14 & \mo 15 & \mo 15\\ \hline
\gr $8$ &  & \mo 1 & \mo 5 & \mo 10 & \mo 15 & \mo 18 & \mo 20 & \mo 21 & \mo 22\\ \hline
\end{tabular}}\vspace{2mm}

\emph{Cf.} \href{http://oeis.org/A026820}
{\texttt{http://oeis.org/A026820}}
\end{minipage}
\end{center}

Por ejemplo, de los dibujos anteriores, solo hay $3 =
p_2(5)$ (los tres primeros) que quepan en dos filas.

Esta tabla, destacada en un libro del siglo XVIII, se conoce
como \emph{tabla de Euler}.\footnote{\textsc{L. Euler}:
  \href{http://eulerarchive.maa.org/pages/E101.html}
       {\emph{Introductio in Analysin Infinitorum I}}
       (1748), pp. 327-328.} La función $p_k(n)$ toma un
valor positivo (es decir: hay alguna partición de $n$ con
$k$ partes o menos) cuando $n \geq 0$ y $k > 0$, además de
en el caso $n = k = 0$. En cualquier otra situación,
definimos $p_k(n) = 0$.

\prop%
{Contando con la capacidad para calcular $p_k(n)$ para
  parámetros cualesquiera, ¿de qué manera se puede obtener
  la cantidad de particiones de un número?}

La complicación del problema tiene como ventaja la
posibilidad de recurrir a la siguiente propiedad:

\begin{center}
{\renewcommand\arraystretch{1.5}
\begin{tabular}
{c|%{M{.7cm}|
M{.7cm}|M{.7cm}|M{.7cm}|M{.7cm}|M{.7cm}|M{.7cm}|M{.7cm}|M{.7cm}|M{.7cm}|}
\diagbox[height=1cm]{\gr $n$}{\gr $k$} &
\gr 0 &\gr 1 &\gr 2 &\gr 3 &\gr 4 &\gr 5 &\gr 6 &\gr 7&\tikzmark{a2}\gr 8\\ \hline
\gr $0$ & \mo 1 & \mo 1& \mo 1& \mo 1& \mo 1& \mo 1& \mo 1& \mo 1& \mo 1\\ \hline
\gr $1$ &  & \mo 1 & \mo 1& \mo 1& \mo 1& \mo 1& \mo 1& \mo 1& \mo 1\\ \hline
\gr $2$ &  & \mo 1 & \mo 2 & \mo 2& \mo 2& \mo 2& \mo 2& \me 2\tikzmark{b2}& \mi\tikzmark{c2} 2\\ \hline
\gr $3$ &  & \mo 1 & \mo 2 & \mo 3 & \me\tikzmark{a1} 3& \mo 3& \mo 3& \mo 3& \mo 3\\ \hline
\gr $4$ &  & \mo 1 & \mo 3 & \mo 4 & \mo 5 & \mo 5& \mo 5& \mo 5& \mo 5\\ \hline
\gr $5$ &  & \mo 1 & \mo 3 & \mo 5 & \mo 6 & \mo 7 & \mo 7& \mo 7& \mo 7\\ \hline
\gr $6$ &  & \mo 1 & \mo 4 & \mo 7 & \mo 9 & \mo 10 & \mo 11 & \mo 11& \mo 11\\ \hline
\gr $7$ &  & \mo 1 & \mo 4 & \me 8\tikzmark{b1} & \mi\tikzmark{c1} 11 & \mo 13 & \mo 14 & \mo 15 & \mo 15\\ \hline
\gr $8$ &  & \mo 1 & \mo 5 & \mo 10 & \mo 15 & \mo 18 & \mo 20 & \mo 21 & \mo 22\\ \hline
\end{tabular}
\tikz[>=stealth, remember picture, overlay, baseline=0 pt]{
\draw[thick, ->, shorten <=3pt, shorten >=6pt]
(pic cs:a1) to[bend right=14pt] (pic cs:c1);
\draw[thick, ->, shorten <=3pt, shorten >=3pt]
(pic cs:b1) to (pic cs:c1);
\draw[thick, ->, shorten <=-1cm, shorten >=8pt]
(pic cs:a2) +(-2mm,0) to[bend right=5pt] (pic cs:c2);
\draw[thick, ->, shorten <=3pt, shorten >=6pt]
(pic cs:b2) to (pic cs:c2);
}}
\end{center}

\begin{resalte}
\centerline{$p_k(n) = p_{k - 1}(n) + p_k(n - k)$.}
\end{resalte}

\prop%
{Explica con tus palabras porque se cumple esta igualdad.}

Esto conduce a un algoritmo recursivo para la enumeración de
las particiones:

\def\pisto{cod_03_02.py}
\def\pista{cod\_03\_02.py}
\begin{minipage}[t]{7cm}
\VerbatimInput%
[frame=lines,framesep=2mm,label=\pista]%
{\cod/\pisto}
\end{minipage}}
\hfill
\begin{minipage}[t]{7cm}
\prop%
{Prueba este algoritmo con distintos valores del parámetro
  para hacerte una idea de su alcance.}
\vspace{5mm}

\prop%
{Compáralo con el algoritmo que usa Sage:}

\begin{Verbatim}[commandchars=\\\{\}]
\sh \inp{sage \ret}

\sg \inp{Partitions(5).cardinality() \ret}
\bl{7}
\sg \inp{Partitions(1000).cardinality() \ret}
\bl{24061467864032622473692149727991}
\sg \inp{Partitions(10 ** 7).cardinality() \ret}
\bl{920271755026...}
\end{Verbatim}

Hay margen entonces para mejorar la solución de la
izquierda.
\end{minipage}

\begin{minipage}[t]{6cm}
Podríamos afinar las reglas de determinación de la función
$p_k(n)$ para ahorrar algunos pasos de la recursión, pero
esto tampoco nos conduce demasiado lejos.
\vspace{1cm}

Date cuenta de que

\[\forall k\geq n\ :\ p_k(n) = p_n(n).\]
\end{minipage}
\hfill
\begin{minipage}[t]{8cm}
\VerbatimInput%
[frame=lines,framesep=2mm]%
{\cod/cod_03_02_v3_add.py}
\end{minipage}
\vspace{5mm}

\prop%
    {
      Traduce los programas anterior a Scheme y da una cota sobre la complejidad espacial y temporal.
    }
    
\prop%
{Siguiendo la estrategia de «memoization», convierte
  el algoritmo recursivo «ingenuo» en dos algoritmos
  (\emph{iterativo/«bottom-up»} y \emph{«top-down»}) para el cálculo de
  las particiones de un número y prográmalos en Scheme.}

\prop%
{Responde a las siguientes preguntas en términos de la complejidad especial y temporal para los nuevos algoritmos.\\[-3mm]

Hasta alcanzar $p(n)$ con la estrategia \emph{bottom-up},
¿cuántos términos de la tabla de Euler se calculan?  Para
cada uno de estos cálculos, ¿a cuántas valores previamente
almacenados se recurre? ¿Cuántas operaciones aritméticas se
emplean en la obtención de $p(n)$?}


Observa que no nos preguntamos por la cantidad de
«operaciones máquina». Para completar el estudio de la
complejidad del algoritmo, sería necesario acotar la talla
de los resultados intermedios. Del resultado indicado al pie
se deduce que esta talla se mantiene acotada por un polinomio.

\prop%
{¿Permite algún ahorro en este caso la estrategia
  \emph{top-down}?}




\vfill

Para comprender el algoritmo tan eficiente que usa
Sage,\footnote{%
\textsc{F. Johansson}:
\href{http://dx.doi.org/10.1112/S1461157012001088}
{«Efficient implementation of the Hardy-Ramanujan-Rademacher
formula»},
LMS J. Comput. Math. {\bf 15}, 2012.}
hace falta sumergirse más en las matemáticas. En concreto, a
partir de la identidad asintótica

\begin{resalte}
[title=Hardy y Ramanujan (1918)]
\[p(n) \sim \frac 1 {4n\sqrt 3}
  \exp(\pi\sqrt{\nicefrac{2n}3}),\]
\end{resalte}

H. Rademacher obtuvo en 1938 una serie que aproxima la
función de particiones y dedujo una cota en $O(\sqrt n)$
para la cantidad de términos que garantiza la obtención del
valor exacto.

%% \sum_{k=1}^N \left(\sqrt{\nicefrac 3 k}\frac 4{24 n - 1}\right)
%% \sum_{h=0}^{k - 1} \delta_{\gcd(h,k), 1}
%% \exp\left(\pi i\left(s(h, k) - \frac{2hn} k\right)\right)
%% \left(\cosh(\frac\pi{6k}\sqrt{24n - 1}) -
%% \frac {6k}{\pi\sqrt{24n - 1}}
%% \sinh(\frac\pi{6k}\sqrt{24n - 1})\right).\]
\end{document}
