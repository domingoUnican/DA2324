% Created 2022-12-08 jue 17:42
% Intended LaTeX compiler: pdflatex
\documentclass{article}
\pagestyle{empty}

\usepackage{amsmath}
\usepackage{mathspec}
\usepackage{nicefrac}
\usepackage{listings}
%fontspec
\setmainfont%[SmallCapsFont={Linux Biolinum Capitals O}]%
{TeX Gyre Pagella}
\def\paragriego{Anaktoria}
% \def\paragriego{GFS Complutum}
\setmonofont{Terminus (TTF)}
%           [BoldFont={TerminusTTF-Bold}]
%\setmathsfont(Digits,Latin,Greek){Euler}
\usepackage[a4paper, text={15.5cm, 25cm}]{geometry}
\addtolength\parskip{3mm}

\usepackage{amssymb}
\usepackage{pagecolor}
\usepackage{pbox}
\usepackage{graphicx}
\usepackage{fancyvrb}
\usepackage[lined,spanish]{algorithm2e}
%\usepackage{minted}
\setlength\parindent{0pt}

\usepackage[hang,flushmargin]{footmisc}

\newcommand\id{\mbox{indeg}}
\newcommand\od{\mbox{outdeg}}

\newcommand\N{\mathbb{N}}
\newcommand\Z{\mathbb{Z}}
\newcommand\Q{\mathbb{Q}}

\newcommand\ret{↵\ } % ⏎, 〈RET〉

\newcommand\tr{^{\mbox{\footnotesize t}}}   % transpuesta

\definecolor{morado}{RGB}{85, 39, 107}
\definecolor{naranja_muy}{RGB}{255, 156, 59} % (1, .612, .231)
\definecolor{naranja_osc}{RGB}{255, 191, 128} % (1, .749, .501)
\definecolor{naranja_claro}{RGB}{255, 229, 204} % (1, .898, .8)
\definecolor{marca}{RGB}{242, 140, 113}

\newcommand\inp[1]{\colorbox{gray!40}{#1}} % ¿Sobra espacio a la izquierda?
\newcommand\mrc{\color{marca!90!black}}

\usepackage{stmaryrd}       % flecha

%% \usepackage{marginnote}
%% \reversemarginpar

\newcounter{pregunta}
\newcommand\prop[1]%
{\addtocounter{pregunta}{1}
\noindent%\marginnote
{\color{naranja_muy}\small\bf\thepregunta)}
%{$\rightarrowtriangle$}\ %
\parbox[t]{.95\linewidth}{\it #1}}

\usepackage[colorlinks=true, urlcolor=blue]{hyperref}

\usepackage{array}
% alineada a la derecha, anchura fija
\newcolumntype{R}[1]{>{\raggedright\arraybackslash}m{#1}}
\newcolumntype{C}{>{$}c<{$}}
\newcolumntype{M}[1]
{>{\centering\arraybackslash $}m{#1}<{$}}

% Para las flechas en tablas...
\usepackage{tikz}
\usetikzlibrary{tikzmark}
\usetikzlibrary{arrows}

\newcommand\aparte[1]{%
\hfill%{\color{black!60}\vline}\quad
\begin{minipage}[t]{.9\linewidth}%
\color{black!80!blue}\small #1
\end{minipage}}

\def\adentro#1{\hspace{.05\linewidth}
  \parbox{.8\linewidth}{\small #1}}

\XeTeXdashbreakstate = 0

\usepackage{xstring}
\usepackage{ifthen}
\usepackage{xifthen}

\usepackage{tcolorbox}
\newtcolorbox{resalte}[1][]%
{coltitle=black,
arc=3pt, colback=naranja_claro,
top=3mm, bottom=3mm, left=3mm, right=3mm,
colframe=naranja_osc, boxrule=1pt,
left skip=0pt, right skip=4cm, #1}
%segmentation style={solid}}
%segmentation engine=empty}
%segmentation style={double=white}}

\newtcolorbox{resalte_chico}[1][]%
{arc=3pt, colback=naranja_claro,
top=1mm, bottom=1mm, left=1mm, right=3mm,
colframe=naranja_osc, boxrule=1pt,
left skip=0pt, right skip=4cm, #1}

\newtcbox{resaltelinea}[1][]%
{tcbox raise base, nobeforeafter,
coltitle=black,
arc=3pt, colback=naranja_claro,
top=1mm, bottom=1mm, left=1mm, right=0pt,
colframe=naranja_osc, boxrule=1pt,
#1}

\newtcbox{resalteajustado}[1][]%
{tcbox raise base, nobeforeafter,
coltitle=black,
arc=3pt, colback=naranja_claro,
top=0mm, bottom=0mm, left=0pt, right=0pt,
colframe=naranja_osc, boxrule=1pt,
#1}

\newcommand\respuestas{{\color{naranja_muy}\bf Respuestas}
\vspace{5mm}}

\newcommand\figs{../figs}
\newcommand\cod{../scripts}
\newcommand\grf{../grafos}

\newcommand\espacio{␣}
\newcommand\vacio{\mbox{\fontspec{FreeSerif}∅}}

\pagecolor{morado!10} % 238, 233, 240 (.933, .914, .941)

\def\truca
{~\par}
\def\trucavuelve
{~\par\vspace{-\baselineskip}}

\usepackage[spanish]{babel}
\usepackage{colortbl}

\newcommand\gr{\color{gray}}

\usepackage{diagbox}
\usepackage{scalerel}
\usepackage{multirow}

\usepackage{eso-pic}

\def\msi{\color{naranja_muy}{\fontspec{FreeSerif}✓}}
\def\mno{{\fontspec{FreeSerif}✗}}

\def\pyt{{\color{naranja_muy}\texttt{>>>}}}
\def\pyc{{\color{naranja_muy}\texttt{...}}}
\def\sh{{\color{naranja_muy}\texttt{\$}}\hspace{-1em} }
\def\shc{{\color{naranja_muy}\texttt{>}}}
\def\sg{{\color{naranja_muy}\texttt{sage:}}}
\def\sgc{{\color{naranja_muy}\texttt{....:}}}
\def\bl#1{{\color{black!50!blue}\texttt{#1}}}
\def\ro#1{{\color{red!80!black}\texttt{#1}}}

\begin{document}
\shorthandoff{>}\shorthandoff{<}

\vspace{5mm}\centerline{\large\bf Práctica de uso de Arduino}\vspace{5mm}
\label{sec:orgbb05154}

El objetivo de esta práctica es recordar todos los conceptos vistos durante el curso y aplicarlo a un caso práctico donde la implementación es compleja.

\textbf{Antes de empezar y aunque sea una práctica en grupo, se pide a todos los integrantes rellenar el siguiente} \href{https://forms.gle/pS8vjtNx2hcGZ7MD9}{formulario}.

La tarea está dirigida a iniciar al
estudiante en el microcontrolador Arduino y consistirá
en que se programe un coche robot para que siga la línea en un circuito (Simulador Oviedo).

\section*{El simulador Oviedo}
Un arduino es un microcontrolador, es decir, un circuito integrado
programable que incluye las tres partes funcionales de un ordenador:
unidad central de procesamiento, memoria y dispositivos de entrada y
salida. Los dispositivos de entrada y salida se les suele denominar \textbf{Pin}.

\begin{figure}[htbp]
\centering
\includegraphics[width=.9\linewidth]{./tabla.png}
\caption{Principales modelos de arduino con sus características: tomada de hetpro-store.com}
\end{figure}

Los pines de un microcontrolador Arduino son de dos
tipos: 
\begin{itemize}
\item Dígitales: La lectura o escritura  una variación  de  voltaje  entre
dos valores  sin  tener en cuenta los intermedios. Por lo
tanto, una señal digital dispone solo de dos estados. Al valor
inferior  de  tensión -Vcc le asociamos un valor lógico LOW o ‘0’,
mientras que al valor superior +Vcc le asociamos HIGH o ‘1’ lógico.
\item Analógicos: En este caso, se puede tomar cualquier magnitud entre
los valores -Vcc y +Vcc.
\end{itemize}



En esta práctica, tenemos un microcontrolador Arduino que, entre otros
elementos electrónicos, consta de: 
\begin{itemize}
\item Motores, que son los que se encargan de mover las ruedas
\item Sensores, que se encargan de detectar el color negro en el suelo.
\end{itemize}

Durante la práctica, utilizaremos los dos tipos de pines. Los
analógicos serán utilizados para mover los motores y los
digitales se utilizarán para las lecturas de los sensores.

Empecemos por los motores, la velocidad de estos es un valor entre $0$
y 180, siendo 
\begin{itemize}
\item $90$ la posición de estar parado
\item $180$ es velocidad máxima hacia delante en el motor derecho y hacia
atrás en el izquierdo
\item $0$ es velocidad máxima hacia delante en el motor  izquierdo y hacia
atrás en el derecho
\end{itemize}

En el siguiente cuadro de código está comentado el uso y
funcionamiento de los motores.


\begin{lstlisting}[frame=lines,fontsize=\scriptsize,linenos]{c}
#include <Servo.h>
Servo servoIzq; // Rueda izquierda
Servo servoDer; // Rueda derecha

int pinServoDer = 9;
int pinServoIzq = 8;
void setup(){
 
  servoIzq.attach(pinServoIzq); // Escribir y leer la velocidad del motor izq
  servoDer.attach(pinServoDer); // Escribir y leer la velocidad del motor dcha
  
}

void stopMotor(){
  servoIzq.write(90); 
  servoDer.write(90);
}

void forwardMotor(){
  servoIzq.write(0); 
  servoDer.write(180);
}
\end{lstlisting}


\prop{Explique con palabras como se hace girar el coche a la izquierda y a la derecha, según el código del fichero \emph{siguelineas.ino}. Diga como se hace para leer los pines de digitales de los sensores, explicando que valores son los que se utilizan para detectar linea en el suelo}
\label{sec:orgf83509a}

Para la programación de un dispositivo Arduino, se utiliza el lenguaje
de programación C, con algunas modificaciones y librerías que se
explican por el funcionamiento del microcontrolador.

La primera modificación es la presencia de dos métodos \textbf{setup} y
\textbf{loop}, y la no presencia del método \textbf{main}. El funcionamiento de un
Arduino es muy parecido al de un \emph{autómata finito} (que se vio en la
asignatura de Lenguajes Formales) y por ello tiene una parte de
configuración (\textbf{setup}) y la siguiente es la parte donde se repite un
mismo procedimiento, a partir de las entradas (\textbf{loop}). Se podría
pensar que este bucle se repite de forma infinita.


Para la programación del coche Arduino se pide utilizar el simulador incluido.


Las instrucciones de uso son las siguientes:
\begin{itemize}
\item Descomprime y ejecuta el simulador mediante main.exe
\item Selecciona la opción “Robot Móvil (4 infrarrojos) y Circuito
\item Selecciona \emph{Archivo->Importar sketch} para cargar el archivo \emph{siguelineas.ino}.
\end{itemize}

Se puede mover el coche Arduino con las teclas de movimiento “AWSD”. Para ejecutar
el archivo, debes desmarcar la acción \emph{Movimiento con el teclado},
pulsar \emph{stop} y \emph{play}  y se
iniciará la ejecución.

\prop{Quite el comentario y complete el código \emph{siguelineas.ino}, de forma que el coche Arduino siga todo el circuito de manera autónoma.}
\label{sec:org15a69fa}

Un problema del programa de \emph{siguelineas.ino} es que si no empieza en
una linea, no hay forma \textbf{determinista} de encontrar el camino de
vuelta a la linea, ya que no se conoce la posición.

\prop{Programe el generador de números pseudoaleatorio «Linear Congruential Generator», busque su definición en Wikipedia y utilice la siguiente estrategia probabilista para encontrar la linea: generar un bit, si el bit es uno girar 90 grados a la derecha y avanzar 400 ms, si es cero girar 90 grados a la izquierda y avanzar. Seguir así hasta encontrar la linea.}
\label{sec:orgbecb0ad}


\end{document}

